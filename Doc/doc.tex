\documentstyle[a4,12pt]{article}

\newcommand\TLambda{$\lambda$}

\title{A Normaliser for Pure and Typed $\lambda$-Calculus\\
       {\normalsize version 2.3}}
\author{Christophe Raffalli
  \thanks{\tt email: Christophe.Raffalli@univ-savoie.fr} \\
  Laboratoire de Math\'ematiques \ LAMA \\ 
  Universit\'e de Savoie} \date{November 1997}

\hfuzz=2pt
\newif\ifnofirst
\newenvironment{definition}{%
  \nofirstfalse%
  \newcommand{\ditem}[1]{\ifnofirst\endlist\endgroup\else\nofirsttrue\fi%
                         {\noindent##1}\begingroup\list{}{}\item}}%
  {\endlist\endgroup}
\def\verbatimfile#1{
  \def\inputfile{\input{#1}}
  \def\beginverb{\begin{verbatim}}
  \expandafter\beginverb\inputfile
}

\begin{document}

\maketitle

\abstract{This document presents a normaliser for pure \TLambda-calculus. It
  can be used for teaching purpose or just to have fun (yes pure
  \TLambda-calculus can be fun~!). It also allows programming in Girard's
  system F. It is reasonably efficient and has enough features to develop
  middle size programs. But be aware, this is not intended to be a useful
  programming language for real purpose implementations~!}

\tableofcontents

\section{Introduction.}

This program provides a useful environment to write programs in pure
\TLambda-calculus or in system F (extended or not with inductive types). Given
a term, the program will compute and print its normal form. This means that,
unlike most functional languages, computation is done under function
abstractions (or lambdas).

This software might be used for teaching \TLambda-calculus or
just to play with it.

It has some interesting features :

\begin{itemize}
\item Three modes of evaluation:
  \begin{description}
  \item[lazy]: call-by-name with some sharing 
  \item[left]: call-by-name with no sharing 
  \item[trace]: call-by-name and printing of each beta-reduction step.
  \end{description}
  
\item It is implemented in Objective Caml (a dialect of ML) so easily portable
  to many machines, even very small ones.

\item It is quite efficient. It uses a High-Order-Abstract-Syntax
  representation of terms which help to benefit from ML management of closure.
  This is specially optimised to give both reasonable performance in speed and
  memory. It's possible to terminate computation of factorial 100 with a
  binary encoding of natural numbers~!
    
\item You can define terms (even using recursive definitions) or include
  files. This gives enough modularity to write fairly large terms.

\item You can use a typing algorithm for system F (with or without inductive
  type) to program more ``safely''.

\end{itemize}

Warning: This documentation does not describe what is the $\lambda$-calculus or
system F. It only describes the syntax and the command used by the normaliser.
To know more about the $\lambda$-calculus or system F, see the bibliography.
 
\section{Getting started.}

The sources are available by {\tt ftp}. They can be compiled very easily
(after you installed Caml-Light and if you have a Unix-like {\tt make}
command).  You can also get a Macintosh stand-alone application which does not
require Caml-Light.

The ftp site to get the program is:
\begin{verbatim}
site = ftp.logique.jussieu.fr
dir  = pub/distrib/raffalli
file = lambda.tar.gz   # Unix version
       lambda.sea.hqx  # Mac version 
\end{verbatim}

To install proceed a follow:
\begin{itemize}
\item On a Unix Machine: get and install Caml-Light (V0.7 only): The ftp site
  is
  \begin{verbatim}
  site = ftp.inria.fr (192.93.2.54)
  dir  = lang/caml-light
  \end{verbatim}
  Go into the directory Src and type {\tt make}. This should produce an
  executable file {\tt lambda} which is the normaliser.  You can also type
  {\tt make lambdaopt}. This should produce an executable file {\tt lambdaopt}
  compiled by the native code compiler (ocamlopt).
  
\item On a Macintosh: The Macintosh version will be available as soon as
  Objective Caml has been ported !

\end{itemize}

You can now launch the program and play with it. Here is the classical
examples of church numerals (What follows the prompt ``{\verb@%>@}'' is what
you should type, everything else is printed by the normaliser):
\begin{verbatim}

---------------------------------
|  The Normaliser, version 2.2  |
| by C. Raffalli, November 1996 |
---------------------------------

Can't read startup file ".lambdarc" (ignored).
%> let 0 = \f x.x;
New term "0" defined.
%> let S = \n f x.f (n f x);
New term "S" defined.
%> let 1 = S 0;
New term "1" defined.
%> let 2 = S 1;
New term "2" defined.
.....
%> let 9 = S 8;
New term "9" defined.
%> let 10 = S 9;
New term "10" defined.
%> let Add = \n m f x.n f (m f x);
New term "Add" defined.
%> let Mul = \n m f.n (m f);
New term "Mul" defined.
%> let 20 = Add 10 10;
New term "20" defined.
.....
%> let 100 = Mul 10 10;
New term "100" defined.
%> let pred = \n.n (\p x y.p (S x) x) (\x y.y) 0 0;
New term "pred" defined.
%> 20;
\f x.f (f (f (f (f (f (f (f (f (f (f (f (f (f (f (f 
(f (f (f (f x)))))))))))))))))))
number of beta-reduction : 66
%> 100 pred 100;
\f x.x
number of beta-reduction : 31010
\end{verbatim}
Note that identifiers can use numbers and also the characters ``{\verb@_@}''
and ``{\verb@'@}'' (but not in first position).

The line ``{\tt Can't read startup file ".lambdarc" (ignored).}'', in the
above example, tells that the program could not find a startup file. Indeed
the program search for a file {\tt .lambdarc} first in your home directory,
and then in the current directory (on macintosh the file {\tt lambdarc}
(without dot) is search in the directory where the application resides).  If
found, the file is included (See {\tt include} command). This file can by used
to set up some path or other initial commands like {\tt recursive}.  On
macintosh, a default startup file is furnished with the distribution, which
setup a path on the {\tt Examples} directory.

Note: You can use Control-C (Command-. on a Mac) to stop looping
commands like the normalisation of a non-normalisable term or the
recursive printing ({\tt prrec} command) of a term using recursive
definitions.

\section{Syntax of term.}

\TLambda-term are constructed using variables, applications and abstractions. 
\begin{description}
\item[Variables]: They can be written using any letters (capital or not),
  digits and (but not in first position) the two characters ``{\verb@_@}'' and
  ``{\verb@'@}'' in any order. Variables are also used as names for defined
  terms (cf {\tt def} and {\tt redef} commands). The following names are
  valid:
\begin{verbatim}
x   y   xx   x0   0   x'   y'   a_variable   a___'' 
\end{verbatim}
  Moreover, the following identifier are {\em keywords} and can not be used as
  variable names:
\begin{verbatim}
  and def fun in include inductive lazy left let margin 
  max_depth max_indent mode off on path print prrec quit 
  rec recursive redef tab trace type typed verbose
\end{verbatim}
  
\item[Application]: is written simply by writing first the function followed by
  its argument: {\tt {\it term1} {\it term2}} is the application of the term
  {\tt\it term1} to the term {\tt\it term2}. Multiple application can be used
  and is left associative: {\tt {\it term1} {\it term2} {\it term3} {\it
      term4} ....} stands for {\tt (...((({\it term1} {\it term2}) {\it
      term3}) {\it term4}) ....)}.
  
\item[Abstraction]: The abstraction of the variable {\verb@x@} in the term
  {\tt\it term} is written {\tt {\verb@\x.@}{\it term}}. Multiple abstraction
  can be used {\tt\verb@\@x1 x2 x3 x4.{\it term}} stands for
  {\tt\verb@\x1.\x2.\x3.\x4.@{\it term}}. \verb@fun@ and \verb@=>@ are
  alternative keywords for \verb@\@ and \verb@.@.

\item[Parenthesis]: Parenthesis may be used for grouping. The scope of an
  abstraction extends as far as possible according to the parenthesis.  So
  {\verb@\x y.x y@} means {\verb@\x.\y.(x y)@} and 
  {\verb@\x y z. x z (y z)@} means {\verb@\x.\y.\z.((x z) (y z))@}.
  
\item[Naming Convention]: When terms are printed by the normaliser, the
  original names of bound variables are kept and, when necessary to avoid
  captures, a number is appended to the name. Thus, if you use meaningful
  names, term can be reasonably readable.

\item[Extensions]: Some useful extensions are provided:
\begin{description}
\item[Local \verb@let@ and \verb@let rec@]: 
  Local definitions in a term are available:
\begin{center}
  \tt let {\it f1 a1 ... an} = {\it term1} and 
          {\it f2 b1 ... bp} = {\it term2} and
          ....
      in {\it term}  
\end{center}
  This \verb@let@ construction define the symbols {\tt\it f1} =
  {\tt\verb@\@{\it a1} ... {\it an}. {\it term1}}, {\tt\it f2} =
  {\tt\verb@\@{\it b1} ... {\it bp}. {\it term2}}, ... local to the definition
  of {\tt\it term}.
  
  It is also possible to use a similar syntax with \verb@let rec@ to have
  local recursive definition. This allows a very simple definition of a
  fix-point: \verb@let rec fix f = f (fix f) in fix@. The use of 
  \verb@let rec@ is only possible when recursive definition are allowed 
  (by default they
  are not allowed). Look at the {\tt let} and {\tt recursive} commands in the
  next section.
  
\item[Side effect]: It is difficult (or even impossible) to read a result of a
  computation printed as a \TLambda-term (even with the naming convention). To
  solve a little this problem, side effect are available. A string like
  \verb@"message"@ is a valid term. It is equivalent to the identity
  (\verb@\x.x@) but when it gets an argument, the string is printed on the
  terminal. Inside string, the following escape sequence are available:
  \begin{center}
  \begin{tabular}{|l|l|}
  \hline
  Sequence & Character denoted \cr \hline
  \verb@\\@ & for one backslash (\verb@\@) \cr
  \verb@\"@ & for double quote (\verb@"@) \cr
  \verb@\n@ & for newline (\verb@CR@)\cr
  \verb@\r@ & for return (\verb@CR@)\cr
  \verb@\t@ & for tabulation (\verb@TAB@)\cr
  \verb@\b@ & for backspace (\verb@BS@)\cr
  \tt\verb@\@{\it ddd} & for the character with ASCII code {\tt\it ddd}
    in decimal\cr
  \hline
  \end{tabular}
  \end{center}
  For instance, the following term prints Church numeral:
\begin{verbatim}
%> let print_church = \n. n "S" "0\n";
New term "print_church" defined.
%> print_church 30;
SSSSSSSSSSSSSSSSSSSSSSSSSSSSSS0
\x.x
number of beta-reduction : 105
\end{verbatim}

\end{description}

\end{description}

\section{The type system.}

Programming in untyped $\lambda$-calculus can be very difficult ! Hopefully
this software provide a type system ! To turn on the type system, you must use
the \verb@typed@ command (see the list of commands). The type system we use is
Girard system F \cite{Kri90e,Gir86,GLT88} with or without inductive types
(inductive types corresponds to fixed-point of types). A description of the
algorithm can be found in \cite{Raf95}. The syntax of types is the following:
\begin{description}
\item[Type variable]: Type variables use the same syntax as term variables.
\item[Parametric type]: If {\tt\it name} is a parametric type with $n$
  arguments defined using
  the {\tt type} command, then {\tt{\it name}[{\it type1},...,{\it typen}]} is
    a valid type if {\tt\it type1}, \dots, {\tt\it typen} are valid types too.
  Defined types with no arguments are used with the same syntax as type
  variables.
\item[Implication]: If {\tt\it type1} and {\tt\it type2} are valid types then 
  {\tt{\it type1} -> {\it type2}} is a valid type. Implications are right
  associative. This mean that \verb@A -> B -> C@ is read as 
  \verb@A -> (B -> C)@.
\item[Quantification]: If {\tt\it name} is a type variable and if {\tt\it
    type} is a type then {\tt\verb@/\@{\it name} {\it type}} is a valid type.
  Quantifications are read with the smallest possible scope: \verb@/\X A -> B@
  is read as \verb@(/\X A) -> B@.
\item[Inductive type]: If {\tt\it name} is a type variable and if {\tt\it type}
  is a type then {\tt\verb@!@{\it name} {\it type}} is a valid type.
  Inductive types are read with the smallest possible scope. This means that
  \verb@!X A -> B@ is read as \verb@(!X A) -> B@. Inductive types are illegal
  by default. To be able to use them, you must use the \verb@inductive@
  command (see the list of commands). If \verb@X@ has a non {\em positive}
  occurrence in \verb@A@, the inductive type \verb@!X A@ is allowed but it is
  not a fixed-point and it is therefore useless.
\end{description}

The typing algorithm is incomplete and do not provide principal types.
Therefore, you need to add some type information in terms. There are two kinds
of type information:
\begin{description}
\item[Type casting]: You can indicate the type of any expression or variables
  using the syntax {\tt{\it term\_or\_var} :{\it type}}. These type
  annotations are read with the smallest possible scope: \verb@t u : A@ is
  read as \verb@t (u : A)@.
\item[Local type variable]: To use a type variable local to a term, you can
  use the syntax {\tt\verb@/\@{\it name} {\it term}}. This make is possible to
  use {\tt\it name} in some type annotation. Be careful the scoping of
  \verb@/\@ in terms is the largest possible like for abstraction while it is
  the smallest possible in types.
\end{description}

There is one important rule concerning type information: they are read from
left to right. This implies that, if you want to understand the reasons of a
type error, you can add some type annotation {\bf before} the end of the piece
of code which could not be typed. Here is an example:
\begin{verbatim}
%> typed;
typed is now on.
%> let delta = \x.x x;
Warning: hiding old definition of "delta".
*** Near:
let delta = \x.x x;
                 ^ 
*** Type error: 
Inferred type is ?4 -> ?3
Used type is ?4
%> let delta = \x:/\X X. x x;
Warning: hiding old definition of "delta".
delta : /\X (/\X1 X1 -> X)
\end{verbatim}

Moreover, the typing algorithm can loop (this is very rare with pure system F,
but quite common when using inductive types). When this appends, you can
interrupt the typing algorithm (using \verb@control-C@) and get an error
message which tell you which part of the code started the loop.

Here are some examples of typed programs:

\subsection*{Booleans:}

\verbatimfile{../Examples/bool.typ}
\end{verbatim}

\subsection*{Pairs:}

\verbatimfile{../Examples/prod.typ}
\end{verbatim}

\subsection*{Church's numeral:}

\verbatimfile{../Examples/church.typ}
\end{verbatim}

\section{List of commands.}

\begin{definition}
\ditem{\tt {\it term} ;} normalise the term
\begin{verbatim}
%> (\x.x) \x.x;
\x.x
number of beta-reduction : 1
\end{verbatim}
   
\ditem{\tt \# ......} comment the end of a line (thus everything
  on the line is ignored after the `{\verb@#@}` character).
\begin{verbatim}
%> # this is a comment~!
\end{verbatim}
  
\ditem{\verb@(* ..... *)@} This is another form of command, which allow you to
comment big part of a file (nested comment are handled properly).
\begin{verbatim}
%> (* this is a comment (* ! *) *)
\end{verbatim}

\ditem{\tt def ....;} Identic to the {\tt let} command (for compatibility
with the previous version).

\ditem{\tt include "{\it filename}" ;} load a file as if you typed its
  content on your terminal.
\begin{verbatim}
%> include "bool";
Opening file "bool".
typed is now off.
New term "True" defined.
New term "False" defined.
New term "if" defined.
New term "an" defined.
New term "or" defined.
New term "not" defined.
Closing file "bool".
\end{verbatim}

\ditem{\tt inductive [on];} 
        turn on the possibility to use inductive types
          with "let" and "redef" commands.
\ditem{\tt inductive off ;}
        turn off the possibility to use inductive types
          with "let" and "redef" commands (this is the default).
\begin{verbatim}
%> type A = !X X;
*** Near:
type A = !X X;
        ^     
*** Syntax error, Inductive types are illegal
%> inductive;
inductive is now on.
%> type A = !X X;
New type A defined.
\end{verbatim}
          
  \ditem{\tt let [rec] {\it name1} = {\it terme1} [and {\it name2} = {\it
      terme2} ...]  ;} define some terms. The definition can be mutually
  recursive (see the command {\tt recursive}) when the keyword {\tt rec} is
  used. Old definition are just hidden, term using the old definition are not
  affected. When terms are printed with the {\tt print} command and when they
  refers to ``hidden'' definition the name are marked with \verb@#old#@ (See
  also the {\tt redef} command).
 
\begin{verbatim}
%> let i = \x.x;
New term "i" defined.
%> let j = \x.x i;
New term "j" defined.
%> let i = \x.x x;
Warning: hiding old definition of "i".
%> print j;
\x.x i#old#
\end{verbatim}
  
  \ditem{\tt margin [{\it num}] ;} Set the maximum number of characters
  the pretty-printer can use on a line. If no number is given, it prints the
  current value. Beware : the {\tt margin} command might affect the value of 
  {\tt max\_indent}.
  
  \ditem{\tt max\_depth [{\it num}] ;} Set the maximum ``depth'' before terms
  are printed as ``...'' by the pretty printer. This depth corresponds
  approximately to the number of imbricated parenthesis. If no number is
  given, it prints the current value.
  
  \ditem{\tt max\_indent [{\it num}] ;} Set the maximum number of indentations
  the pretty printer can use. If no number is given, it prints the current
  value.
        
\ditem{\tt mode {\it amode} ;} change the evaluation mode. {\tt\it amode}
  can be {\tt lazy}, {\tt left} or {\tt trace}.

\ditem{\tt mode ;} print the current mode.
\begin{verbatim}
%> mode left;
Evaluation mode changed to "left".
%> mode;
Evaluation mode is "left".
\end{verbatim}

\ditem{\tt path "{\it directory}" ;}
        add "directory" to the path-list used by the "include" command

\ditem{path ;}
        prints the current path-list
\begin{verbatim}
%> path "../Examples";
"../Examples" added to the path-list
%> path;
The path-list is:
    ../Examples
    /users/cs/raffalli/camllight/Normaliser/Examples
\end{verbatim}
        
\ditem{\tt print {\it term} ;}
        print the term (no normalisation).

\ditem{\tt prrec {\it term} ;}
        print the term (no normalisation) and expend all definitions.
        This does not terminate if the {\tt\it term} uses recursive definition
        (See the {\tt recursive} command). However term using the fix-point
        combinator ``\verb@!@'' are printed correctly. 
\begin{verbatim}
%> let i = \x.x;
New term "i" defined.
%> let j = \x.i x;
New term "j" defined.
%> print j;
\x.i x
%> prrec j;
\x.(\x1.x1) x
%> j;
\x.x
number of beta-reduction : 1
\end{verbatim}
          
\ditem{\tt quit ;}
        quit the program.
\begin{verbatim}
%> quit;
Bye.
\end{verbatim}
        
\ditem{\tt recursive [on];} 
        turn on the possibility to use recursive definition 
          with "let" and "redef" commands.
\ditem{\tt recursive off ;}
        turn off the possibility to use recursive definition 
          with "let" and "redef" commands (this is the default).
\begin{verbatim}
%> let fix = \f.fix f;

*** Unbound identifier: fix
%>  recursive;
Recursive definition are now allowed.
%> let fix = \f.fix f;
New term "fix" defined.
\end{verbatim}
          
\ditem{\tt redef [rec] {\it name1} = {\it terme1} [and {\it name2} = {\it
              terme2} ...] ;} same as the command {\tt let} but old
          definition are replaced by the new one, term using the old
          definition now use the new one.
\begin{verbatim}
%> let i = \x.x;
New term "i" defined.
%> let j = \x.i x;
New term "j" defined.
%> j;
\x.x
number of beta-reduction : 1
%> redef i = \x.x x;
Warning: redefinition of "i".
%> j;
\x.x x
number of beta-reduction : 1
\end{verbatim}

\ditem{\tt tab [{\it num}] ;}
        Set the size of the tabulations used by the pretty printer. If no size
        is given, it prints the current value.

\begin{verbatim}
%> tab;
tab = 2;
%> tab 5;
tab = 5
%> print Euclide;
\n m.n
     (\np.Euclide np m
          (\q r.catch (Sub (nSz r) m) (\nr.pair (So q) nr)
          (pair (nSz q) (nSz r))))
     (\np.Euclide np m
          (\q r.catch (Sub (So r) m) (\nr.pair (So q) nr)
          (pair (nSz q) (So r))))
     (pair End End):(Bin -> Bin -> P[Bin,Bin])
\end{verbatim}
        
\ditem{\tt trace {\it name} [{\it nbargs}] ;} Ask the normaliser to
        print the arguments given to the term named {\it name}. If the
        optional argument {\it nbargs} is specified, at most {\it nbargs}
        arguments are printed.  If other definitions where traced they remain
        traced however.  In trace mode only the beta-reductions are traced (not
        the function call specify by the \verb!trace! command.

\ditem{\tt trace {\it name} off ;}
        Turn off tracing of the term named "name".

\ditem{\tt trace ;}
        Print informations about traced definitions.

\begin{verbatim}
%> let i = \x.x;
New term "i" defined.
%> trace i 1;
"i" is now traced (at most 1 argument is printed).
All beta-reduction are no longer traced.
%> i i i;
 ....
    i is applied to:
        i
 ....
    i is applied to:
        i
 ....
    i is applied to nothing.
\x.x
number of beta-reduction : 2
%> trace i off;
"i" is no longer traced.
%> i i i;
\x.x
number of beta-reduction : 2
%> mode trace;
Evaluation mode changed to "trace".
%> i i i;
 .....
    (\x.x)
        i
        i

 .....
    (\x.x)
        i

\x.x
number of beta-reduction : 2
\end{verbatim}

\ditem{\tt type {\it name} = {\it a\_type};}
  Define a new type abbreviation of name {\tt\it name}.
\ditem{\tt type {\it name}[{\it name1},...,{\it namen}] = {\it a\_type};}
Define a new type abbreviation of name {\tt\it name} using $n$ parameters.
\begin{verbatim}
%> type Bool = /\X (X -> X -> X);
New type Bool defined.
%> type Pair[A,B] = /\X((A -> B -> X) -> X);
New type Pair defined.
\end{verbatim}

\ditem{\tt typed [on];} 
        turn on the type system.
\ditem{\tt typed off ;}
        turn off the type system.
\begin{verbatim}
%> \x.x;
_ = \x.x
number of beta-reduction : 0
%> typed;
typed is now on.
%> \x.x;
_ : /\X (X -> X)
_ = \x.x
number of beta-reduction : 0
\end{verbatim}

\end{definition}
      
\section{Lazy evaluation and the representation of data.}

To make possible evaluation of fairly large terms, a lazy normalisation is
provided. This means that in an application \verb@t u@, all the reduction on
the argument \verb@u@ will be done only once till \verb@u@ is reduced to an
abstraction or a variable apply to some arguments. Reduction under an
abstraction won't be shared. 

This means that not all the coding of data in pure \TLambda-calculus will be
efficient in lazy evaluation. A simple rule to follow is that all constructors
should have a form \verb@\a1 .. ap x1 ... xn. xi a1 ... ap@. In this term,
{\tt a1 ... an} are the arguments of the constructor while {\tt x1 ... xn} are
used to distinguish this constructor (identified by the head variable {\tt xi})
from others.  If you respect this rule, data will only be computed once.

Note that Church representation of natural numbers is not good with respect to
lazy evaluation has successor is \verb@\n f x. f (n f x)@ (the alternative
definition \verb@\n f x. n f (f x)@ is not better). Indeed, if you compute a
non null Church numeral {\tt t} only the reduction to the form \verb@\f x.f
t'@ is shared, but no further reduction on \verb@t'@. A better unary
representation of natural numbers are recursive natural (in this case,
\verb@t'@ is a closed term and further reduction are shared):
\begin{verbatim}
let 0 = \f x. x;
let S = \n f x. f n;
let Pred = \n. n (\x.x) 0;
\end{verbatim}

If you only use this kind of coding for your data-types, you should get
performance proportional to those you would get with any other functional
language~!

\section{Using the trace mode.}

The trace mode prints the result as it is being computed on an unindented
line. When it must print some tracing information, then it prints some dots,
and prints the information indented and it goes on till the end of the
normalisation. 

For instance if we consider the reduction of \verb@(\x y.x y (x x) (y x))
\z.z@ when all beta-reductions are printed:
\begin{verbatim}
%> mode trace;
Evaluation mode changed to "trace".
%> (\x y.x y (x x) (y x)) \z.z;
 .....                                  1 2 3 4
    (\x y.x y (x x) (y x))              1
        (\z.z)                          1

\y. .....                               2 3 4
    (\z.z)                              2 
        y                               2
        ((\z.z) (\z.z))                 2  
        (y (\z.z))                      2

y  .....                                3 4
    ((\z.z)                             3 
        (\z.z))                         3
     (y (\z.z))                         3

(\z.z) (y (\z1.z1))                     4
number of beta-reduction : 3
\end{verbatim}

We added on the right some numbers to mark the lines (they are not printed by
the normaliser). If you read everything on the line marked by {\tt 1}, you
get the state of the term before the first reduction. If you read everything
on the lines marked by {\tt 2}, you get the state of the term before the
second reduction and so on. If you only read the unindented line (marked by
{\tt 4}), you read the normal form of the term: \verb@\y. y (\z.z) (y
(\z1.z1))@.

When tracing some functions (in lazy or left mode), the printed information is
just the name of the traced function and the arguments it is receiving (or
only some of them):

\begin{verbatim}
%> let i = \z.z;
New term "i" defined.
%> trace i;
"i" is now traced.
%> (\x y.x y (x x) (y x)) i;
_ = ....
    i is applied to:
        y
        (i i)
        (y i)
 ....
    i is applied to:
        i
 ....
    i is applied to nothing.
 ....
    i is applied to nothing.

\y.y (\z.z) (y (\z.z))
number of beta-reduction : 3
\end{verbatim}

You should always keep in mind that tracing or debugging can be very difficult
because \TLambda-terms are not very readable and because call-by-name produces
an unintuitive order of evaluation.  However, the system try to keep the
original names of bound variables and this can help when debugging if you
choose meaningful names for bound variables.

\bibliography{doc}
\bibliographystyle{plain}

\nocite{Bar84}
\nocite{Kri90}
\nocite{Roz93}
\nocite{GLT88}

\end{document}

